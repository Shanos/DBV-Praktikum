\documentclass[a4paper,11pt,oneside]{scrartcl}
\parindent 0pt
\usepackage[utf8]{inputenc}
\usepackage{amsmath}
\usepackage{amssymb}

%opening
\title{Digitale Bildverarbeitung - Lösung Blatt 3 (Fourier-Transf., Splines, DFT)}
\author{Thomas Waldecker\\Stefan Giggenbach}

\begin{document}

\maketitle

\newpage

\section*{Aufgabe 1: Fourier-Beziehungen}
\subsection*{1a) Verschiebung im Ortsbereich}

$\mathcal{F}\{f(x-\alpha,y-\beta)\}=\iint_{-\infty}^{\infty}f(x-\alpha,y-\beta)\cdot e^{-j2\pi(xu+yv)} dx dy$ \\

Mit den Substitutionen $\xi=x-\alpha$ ($x=\xi+\alpha$) und $\eta=y-\beta$ ($y=\eta+\beta$) ergibt sich \\

$=\iint_{-\infty}^{\infty}f(\xi,\eta)\cdot e^{-j2\pi((\xi+\alpha)u+(\eta+\beta)v)} d\xi d\eta$ \\

Durch Ausklammern der konstanten Faktoren erhält man \\

$=e^{-j2\pi(\alpha u+\beta v)}\iint_{-\infty}^{\infty}f(\xi,\eta)\cdot e^{-j2\pi(\xi u+\eta v)} d\xi d\eta$ \\

Da das Integral der Fouriertransformation $\mathcal{F}\{f(\xi,\eta)\}=F(u,v)$ entspricht folgt \\

$\mathcal{F}\{f(x-\alpha,y-\beta)\}=F(u,v)\cdot e^{-j2\pi(\alpha u+\beta v)}$

\subsection*{1b) Faltungssatz}

$\mathcal{F}\{h(x,y)\ast f(x,y)\}=\iint_{-\infty}^{\infty}[\iint_{-\infty}^{\infty}h(\chi,\psi)\cdot f(x-\chi,y-\psi) d\chi d\psi]\cdot e^{-j2\pi(xu+yv)} dx dy$ \\

Durch Ausklammern und Umstellen der konstanten Faktoren erhält man \\

$=\iint_{-\infty}^{\infty}h(\chi,\psi)\cdot[\iint_{-\infty}^{\infty}f(x-\chi,y-\psi)\cdot e^{-j2\pi(xu+yv)} dx dy] d\chi d\psi$ \\

Da das innere Integral mit der Beziehung aus Teilaufgabe 1a) der Fouriertransformation $\mathcal{F}\{f(x-\chi,y-\psi)\}=F(u,v)\cdot e^{-j2\pi(\chi u+\psi v)}$ entspricht folgt \\

$=\iint_{-\infty}^{\infty}h(\chi,\psi)\cdot F(u,v)\cdot e^{-j2\pi(\chi u+\psi v)} d\chi d\psi$ \\

Durch Ausklammern des konstanten Faktor erhält man \\

$=F(u,v)\iint_{-\infty}^{\infty}h(\chi,\psi)\cdot e^{-j2\pi(\chi u+\psi v)} d\chi d\psi$ \\

Da das Integral der Fouriertransformation $\mathcal{F}\{h(\chi,\psi)\}=H(u,v)$ entspricht folgt \\

$\mathcal{F}\{h(x,y)\ast f(x,y)\}=F(u,v)\cdot H(u,v)$

\newpage

\section*{Aufgabe 2: Approximation von sinc(x) durch kubischen Spline}

\begin{math}
\begin{array}{ll}
f(x) & = 
 \left\{ 
  \begin{array}{l}
   f_1(x) = a_1x^3 + b_1x^2 + c_1x + d_1\\
   f_2(x) = a_2x^3 + b_2x^2 + c_2x + d_2\\
   f_3(x) = 0\\
  \end{array} 
 \right.
\\[0.5cm]
f'(x) & = 
 \left\{ 
  \begin{array}{l}
   f_1'(x) = 3a_1x^2 + 2b_1x + c_1 \\
   f_2'(x) = 3a_2x^2 + 2b_2x + c_2 \\
   f_3'(x) = 0\\
  \end{array} 
 \right.
\\[0.5cm]
f''(x) & = 
 \left\{
  \begin{array}{l}
   f_1''(x) = 6a_1x + 2b_1\\
   f_2''(x) = 6a_2x + 2b_2\\
   f_3''(x) = 0\\
  \end{array}
 \right.
 \end{array}
\end{math}

$f_1(0) = 1 = a_1 \cdot 0 + b_1 \cdot 0 + c_1 \cdot 0 + d_1 = d_1 = 1$\\
$f_1'(0) = 0 = 3a_1 \cdot 0 + 2b_1 \cdot 0 + c_1 = c_1 = 0$\\

\newpage

\section*{Aufgabe 3: DFT/Abtastung/Abtasttheorem}
\subsection*{a)}

$\Delta x=\frac{L}{N}=\frac{10\,mm}{256}=0.0391\,mm$

\subsection*{b)}

$u_{a}=\frac{N}{L}=\frac{256}{10\,mm}=25.6\,\frac{1}{mm}$

\subsection*{c)}

$\Delta u=\frac{1}{N\Delta x}=\frac{1}{256\cdot0.0391\,mm}= 0.1\,\frac{1}{mm}$

\subsection*{d)}

Die Schwingung $\sin(16\pi k)$ besitzt eine Kreisfrequenz von $\omega=16\pi\,\frac{1}{mm}$ mit $\omega=2\pi f$ also eine Frequenz von $f=\frac{16\pi}{2\pi}\,\frac{1}{mm}=8\,\frac{1}{mm}$. Zusammen mit dem Intervall $\Delta u$ aus Tailaufgabe c) ergeben sich \\

$l_{1}=\frac{f}{\Delta u}=\frac{8\,\frac{1}{mm}}{0.1\,\frac{1}{mm}}=80$ und $l_{2}=256-80=176$

\subsection*{e)}

$f_{max}=\frac{u_{a}}{2}=12.8\,\frac{1}{mm}$

\end{document}
