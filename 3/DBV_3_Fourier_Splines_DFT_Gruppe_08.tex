\documentclass[a4paper,10pt]{scrartcl}
\usepackage[utf8]{inputenc}
\usepackage{amsmath}

%opening
\title{Digitale Bildverarbeitung - Lösung Blatt 3 (Fourier-Transf., Splines, DFT)}
\author{Thomas Waldecker\\
	Stefan Giggenbach}

\begin{document}

\maketitle

\section{Aufgabe 1: Fourier-Beziehungen}
\subsection{1a) Verschiebung im Ortsbereich}

$\{f(x-\alpha,y-\beta)\}=\iint_{1}^{5}{f(x-\alpha,y-\beta)\cdot e^{-j2\pi(xu+yv)}}$



\section{Aufgabe 2: Spline}

\begin{displaymath}
f(x) = 
 \left\{ 
  \begin{array}{l}
   f_1(x) = a_1x^3 + b_1x^2 + c_1x + d_1\\
   f_2(x) = a_2x^3 + b_2x^2 + c_2x + d_2\\
   f_3(x) = 0\\
  \end{array} 
   \right.
\end{displaymath}

$f_1'(x) = 3a_1x^2 + 2b_1x + c_1 $\\
$f_2'(x) = 3a_2x^2 + 2b_2x + c_2 $\\

$f_1''(x) = 6a_1x + 2b_1$\\
$f_2''(x) = 6a_2x + 2b_2$\\

$f_1(0) = 1 = a_1 \cdot 0 + b_1 \cdot 0 + c_1 \cdot 0 + d_1 = d_1 = 1$\\
$f_1'(0) = 0 = 3a_1 \cdot 0 + 2b_1 \cdot 0 + c_1 = c_1 = 0$\\


\end{document}
